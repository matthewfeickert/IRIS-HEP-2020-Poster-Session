% a0poster Portrait Poster
% LaTeX Template
% Version 1.0 (22/06/13)
%
% The a0poster class was created by:
% Gerlinde Kettl and Matthias Weiser (tex@kettl.de)
%
% This template has been downloaded from:
% http://www.LaTeXTemplates.com
%
% License:
% CC BY-NC-SA 3.0 (http://creativecommons.org/licenses/by-nc-sa/3.0/)
%

%----------------------------------------------------------------------------------------
%	PACKAGES AND OTHER DOCUMENT CONFIGURATIONS
%----------------------------------------------------------------------------------------

% \directlua{pdf.setminorversion(7)}
\documentclass[princeton,portrait]{a0poster}

\usepackage{multicol} % This is so we can have multiple columns of text side-by-side
\columnsep=100pt % This is the amount of white space between the columns in the poster
% \columnseprule=3pt % This is the thickness of the black line between the columns in the poster

\usepackage[svgnames]{xcolor} % Specify colors by their 'svgnames', for a full list of all colors available see here: http://www.latextemplates.com/svgnames-colors

\usepackage{times} % Use the times font
%\usepackage{palatino} % Uncomment to use the Palatino font

\usepackage{graphicx} % Required for including images
\graphicspath{{figures/}} % Location of the graphics files
\usepackage{booktabs} % Top and bottom rules for table
\usepackage[font=small,labelfont=bf]{caption} % Required for specifying captions to tables and figures
\usepackage{amsfonts, amsmath, amsthm, amssymb} % For math fonts, symbols and environments
\usepackage[english]{babel}
\usepackage[T1]{fontenc}
\usepackage[utf8]{inputenc}
\usepackage{wrapfig} % Allows wrapping text around tables and figures
\usepackage{bm}
\usepackage{tikz}
\usepackage[skins]{tcolorbox}
\usepackage{fancyhdr}
\newlength{\marginoffset}
\setlength{\marginoffset}{3.25cm}
\usepackage[%
left=\marginoffset,%
right=\marginoffset,%
top=0.5cm,%
bottom=0cm]{geometry}
\usepackage[hidelinks]{hyperref}

\newcommand\crule[3][black]{\textcolor{#1}{\rule{#2}{#3}}}

\definecolor{NSF-blue}{HTML}{3487c0}
\definecolor{NSF-violet}{HTML}{8b82bb}

\begin{document}
%
%----------------------------------------------------------------------------------------
%	POSTER HEADER
%----------------------------------------------------------------------------------------
%
% The header is divided into two boxes:
% The first is 75% wide and houses the title, subtitle, names, university/organization and contact information
% The second is 25% wide and houses a logo for your university/organization or a photo of you
% The widths of these boxes can be easily edited to accommodate your content as you see fit
%
% \begin{minipage}[b]{0.75\linewidth}
% Header
\begin{minipage}[b]{\linewidth}
 \hspace{-\marginoffset}
 % IRIS-HEP logo
 \begin{minipage}{0.25\linewidth}
  \href{https://iris-hep.org/}{\includegraphics[width=\linewidth]{IRIS-HEP_logo.eps}}
 \end{minipage}%
 \quad
 % project title
 \begin{minipage}{0.54\linewidth}
  \VERYHuge \color{Black} \textbf{pyhf}\\[0.5cm] \color{Black}\Huge{pure Python implementation of HistFactory}\\[1cm] % Subtitle
  \color{DarkSlateGray} % DarkSlateGray color for the rest of the content
  \LARGE {\href{https://www.matthewfeickert.com/}{\underline{Matthew Feickert}$^{1}$}}, \href{http://www.lukasheinrich.com/}{Lukas Heinrich$^{2}$}, \href{https://giordonstark.com/}{Giordon Stark$^{3}$}\\[0.5cm] % Author(s)
  \normalsize {1 University of Illinois at Urbana-Champaign,~2 CERN,~3 University of California Santa Cruz}\\[1cm]% University/organization
  \color{Black}
 \end{minipage}
 % pyhf logo
 \begin{minipage}{0.18\linewidth}
  \vspace{-2.5cm}
  \begin{center}
   \href{https://github.com/scikit-hep/pyhf}{\includegraphics[width=\linewidth]{pyhf-logo.png}}
  \end{center}

  \vspace{-0.5cm}
  \begin{center}
   \href{https://doi.org/10.5281/zenodo.1169739}{\includegraphics[width=\linewidth]{zenodo_doi.pdf}}
  \end{center}
 \end{minipage}%
 % GitHub stuff
 \begin{minipage}{0.18\linewidth}
  \begin{minipage}{0.25\linewidth}
   \begin{flushleft}
    \href{https://github.com/scikit-hep/pyhf}{\includegraphics[width=\linewidth]{GitHub_logo.png}}
   \end{flushleft}
  \end{minipage}%
  \,\!
  \begin{minipage}{0.25\linewidth}
   \begin{flushleft}
    \href{https://github.com/scikit-hep/pyhf}{\includegraphics[width=\linewidth]{github_qr_code.pdf}}
   \end{flushleft}
  \end{minipage}%
  %
 \end{minipage}%
 \vspace{-1cm}
\end{minipage}
\noindent\makebox[\linewidth]{\crule[NSF-blue]{\paperheight}{0.5cm}}

%----------------------------------------------------------------------------------------

\begin{multicols}{2} % This is how many columns your poster will be broken into, a portrait poster is generally split into 2 columns

 %----------------------------------------------------------------------------------------
 %	ABSTRACT
 %----------------------------------------------------------------------------------------

 % \color{Navy} % Navy color for the abstract

 %----------------------------------------------------------------------------------------
 %	INTRODUCTION
 %----------------------------------------------------------------------------------------

 % \color{SaddleBrown} % SaddleBrown color for the introduction

 \section*{\LARGE\color{MediumBlue} HistFactory}
 \large
 One of the most widely used statistical models in \textbf{high energy physics} for binned measurements and searches

 \begin{center}
  \includegraphics[width=\linewidth]{HistFactory_result_examples.png}
 \end{center}
 %
 \begin{minipage}{0.33\linewidth}
  \begin{flushleft}
   \large\quad~~\textbf{Standard Model}
  \end{flushleft}
 \end{minipage}%
 \quad
 \begin{minipage}{0.33\linewidth}
  \begin{flushleft}
   \large\quad~\textbf{Supersymmetry}
  \end{flushleft}
 \end{minipage}%
 \quad
 \begin{minipage}{0.33\linewidth}
  \begin{flushleft}
   \large\qquad~~\textbf{Exotics}
  \end{flushleft}
 \end{minipage}%

 \section*{\LARGE\color{MediumBlue} Declarative binned likelihoods}

 \[
  f(\bm{n}, \bm{a} \,|\,\bm{\phi},\bm{\chi}) = \underbrace{\color{blue}{\prod_{c\in\mathrm{\,channels}} \prod_{b \in \mathrm{\,bins}_c}\textrm{Pois}\left(n_{cb} \,\middle|\, \nu_{cb}\left(\bm{\eta},\bm{\chi}\right)\right)}}_{\substack{\text{Simultaneous measurement}\\%
    \text{of multiple channels}}} \underbrace{\color{red}{\prod_{\chi \in \bm{\chi}} c_{\chi}(a_{\chi} |\, \chi)}}_{\substack{\text{constraint terms}\\%
    \text{for }\text{``auxiliary measurements''}}}
 \]

 \noindent\textcolor{blue}{Primary Measurement}:
 \begin{itemize}
  \item Multiple disjoint ``channels'' (e.g. event observables) each with multiple bins of data
  \item Example parameter of interest: strength of physics signal, $\mu$
 \end{itemize}
 \textcolor{red}{Auxiliary Measurements}:
 \begin{itemize}
  \item Nuisance parameters (e.g. in-situ measurements of background samples)
  \item Systematic uncertainties (e.g. normalization, shape, luminosity)
 \end{itemize}

 \section*{\LARGE\color{MediumBlue} Performance}
 Efficient use of tensor computation makes pyhf fast
 \begin{center}
  \includegraphics[width=\linewidth]{performance_only.pdf}
 \end{center}
 Competitive with traditional \texttt{C++} implementation --- often faster

 \section*{\LARGE\color{MediumBlue} Hardware Acceleration}
 For machine-learning-library tensor backends the computational graph can be transparently placed on hardware accelerators: \textbf{GPUs} and \textbf{TPUs} for order of magnitude speed-up in computation
 \begin{center}
  \includegraphics[width=\linewidth]{scaling_hardware.pdf}
 \end{center}

 \section*{\LARGE\color{MediumBlue} Automatic Differentiation}

 Tensor libraries from machine learning frameworks provide exact gradients of computational graphs (likelihood). Useful for minimization!

 \section*{\LARGE\color{MediumBlue} Optimizers}

 Leverage \href{https://docs.scipy.org/doc/scipy/reference/generated/scipy.optimize.minimize.html}{\texttt{scipy.optimize.minimize}} and MINUIT across all backends.
 More advanced optimizers to come soon!

 \section*{\LARGE\color{MediumBlue} Implementation}
 \begin{center}
  % \includegraphics[width=0.8\linewidth]{computational_graph3.pdf}
  % \includegraphics[width=0.8\linewidth]{computational_graph.png}
  \includegraphics[width=0.6\linewidth]{computational_graph.png}
 \end{center}
 The computational graph of multidimensional array operations for likelihood function of a physics analysis defined through HistFactory

 \vspace{1cm}
 %
 \begin{minipage}{0.25\linewidth}
  \begin{center}
   \href{https://github.com/numpy/numpy}{\includegraphics[width=\linewidth]{NumPy_logo.pdf}}
  \end{center}
 \end{minipage}%
 \quad
 \begin{minipage}{0.25\linewidth}
  \begin{center}
   \href{https://github.com/tensorflow/tensorflow}{\includegraphics[width=\linewidth]{TensorFlow_logo.pdf}}
  \end{center}
 \end{minipage}%
 \quad
 \begin{minipage}{0.25\linewidth}
  \begin{center}
   \href{https://github.com/pytorch/pytorch}{\includegraphics[width=0.85\linewidth]{Pytorch_logo.pdf}}
  \end{center}
 \end{minipage}%
 \quad
 \begin{minipage}{0.25\linewidth}
  \begin{flushleft}
   \href{https://github.com/google/jax}{\includegraphics[width=0.5\linewidth]{JAX_logo.png}}
  \end{flushleft}
 \end{minipage}%
 \vspace{1cm}

 \noindent Use of $n$-dimensional array (``tensor'') operations through a common API layer around high performance tensor libraries

 \section*{\LARGE\color{MediumBlue} JSON Specification}
 The \textcolor{blue}{full likelihood} can be expressed as a \textbf{single JSON document}.
 Archive friendly for analysis presentation!
 \vspace{0.5em}
 \begin{center}
  \href{https://raw.githubusercontent.com/scikit-hep/pyhf/master/docs/examples/json/2-bin_1-channel.json}{\includegraphics[width=0.5\linewidth]{carbon_JSON_spec_annotated.png}}

  {\small\textbf{Example:} 2 binned single channel with 2 samples with 1 parameter of interest and 1 nuisance parameter}
 \end{center}
 \noindent Used by ATLAS to \textcolor{blue}{reproduce} and \textcolor{blue}{publish likelihoods} to HEPData\\(\textcolor{blue}{\href{https://atlas.web.cern.ch/Atlas/GROUPS/PHYSICS/PUBNOTES/ATL-PHYS-PUB-2019-029/}{ATL-PHYS-PUB-2019-029}})

 \begin{minipage}{\linewidth}
  \begin{minipage}{0.5\linewidth}
   \begin{center}
    \href{https://atlas.web.cern.ch/Atlas/GROUPS/PHYSICS/PUBNOTES/ATL-PHYS-PUB-2019-029/}{\includegraphics[width=\linewidth]{ATL-PHYS-PUB-2019-029_multi-b_limit.pdf}}
   \end{center}
  \end{minipage}
  \begin{minipage}{0.5\linewidth}
   \begin{center}
    % \includegraphics[width=\linewidth]{CERN_new_story.png}
    \href{https://home.cern/news/news/knowledge-sharing/new-open-release-allows-theorists-explore-lhc-data-new-way}{\includegraphics[width=0.68\linewidth]{CERN_new_story_crop.png}}
   \end{center}
  \end{minipage}
 \end{minipage}
 % \begin{center}
 %  % \href{https://raw.githubusercontent.com/scikit-hep/pyhf/master/docs/examples/json/2-bin_1-channel.json}{\includegraphics[width=0.7\linewidth]{carbon_JSON_spec_annotated.png}}
 %  \href{https://raw.githubusercontent.com/scikit-hep/pyhf/master/docs/examples/json/2-bin_1-channel.json}{\includegraphics[width=0.55\linewidth]{ATL-PHYS-PUB-2019-029_multi-b_limit.pdf}}
 % \end{center}
 % \vspace{-1em}
 % \begin{center}
 %  {\small\textbf{Example:} 2 binned single channel with 2 samples with 1 parameter of interest and 1 nuisance parameter}
 % \end{center}
 % \begin{center}
 %  % \includegraphics[width=0.7\linewidth]{carbon_pyhf_CLs.png}
 %  \includegraphics[width=0.55\linewidth]{CERN_new_story.png}
 %  % \includegraphics[width=0.5\linewidth]{carbon_pyhf_CLs.png}
 % \end{center}
 %
 % \begin{center}
 %  JSON Patch standard allows for reinterpretation with \textcolor{blue}{new signal models}
 % \end{center}
 % \begin{center}
 %  % \includegraphics[width=0.7\linewidth]{carbon_JSON_patch.png}
 %  \includegraphics[width=0.5\linewidth]{carbon_JSON_patch.png}
 % \end{center}
 %
 % \begin{center}
 %  % \includegraphics[width=0.7\linewidth]{carbon_reinterpretation.png}
 %  \includegraphics[width=0.5\linewidth]{carbon_reinterpretation.png}
 % \end{center}
\end{multicols}
%
\vfill
\noindent\makebox[\linewidth]{
 \noindent\fcolorbox{NSF-violet}{NSF-violet}{%
  \begin{minipage}[t]{\pagewidth}
   \begin{minipage}{0.1\linewidth}
    \begin{center}
     % Use height from template
     \href{https://www.nsf.gov/}{\includegraphics[height=2.19in]{NSF_logo.png}}
    \end{center}
   \end{minipage}%
   \qquad
   \begin{minipage}{0.8\linewidth}
    {\large%
     \noindent\textcolor{white}{%
      \href{https://www.nsf.gov/awardsearch/showAward?AWD_ID=1836650}{%
       \noindent This project is supported by National Science Foundation under Cooperative Agreement OAC-1836650.
       Any opinions, findings, conclusions or recommendations expressed in this material are those of the authors and do not necessarily reflect the views of the National Science Foundation.​}%
     }
    }
   \end{minipage}%
   \begin{minipage}{0.05\linewidth}~\end{minipage}%
  \end{minipage}}%
}% end makebox
\end{document}
